\documentclass[a4paper]{article}
\usepackage{packages}
\usepackage{cite}
\usepackage{hyperref}
\hypersetup{
    pdftoolbar=true,
    pdfmenubar=true,
    pdffitwindow=false,
    pdfstartview={FitH},
    pdftitle={TITOLO },
    pdfauthor={AUTORI},
    pdfsubject={ARGOMENTO},
    pdfkeywords={},
    pdfnewwindow=true,
    colorlinks=true, %set this false for printable version
    linkcolor=blue,
    citecolor=blue,
    urlcolor=blue,
}

\theoremstyle{plain}\newtheorem{teo}{Teorema}[section]
\theoremstyle{plain}\newtheorem{prop}[teo]{Proposição}
\theoremstyle{plain}\newtheorem{lem}[teo]{Lema}
\theoremstyle{plain}\newtheorem{cor}[teo]{Corolário}
\theoremstyle{definition}\newtheorem{defi}[teo]{Definição}
\theoremstyle{remark}\newtheorem{rem}[teo]{Observação}
\theoremstyle{definition}\newtheorem{example}[teo]{Exemplo}
\theoremstyle{remark}\newtheorem{step}{\bf Step}
\title{Introdução à Teoria dos Conjuntos - Prova 2}
\author{Ariel Serranoni Soares da Silva  - Número USP: 7658024\\
Pedro Felizatto - Número USP:9794531\\
Pietro Mesquita Piccione - Número USP: 4630640\\
Mateus Schmidt Mattos Lopes Pereira - Número USP: 10262892\\
Luís Cardoso - Número USP: 4552403}
\date{\today}
\begin{document}
\maketitle
\section*{Observações iniciais}

As notas de aula a seguir foram produzidas com base na Seção 3 do Capítulo 12 de
\cite{jech}. A ideia do nosso trabalho é seguir fielmente o conteúdo abordado no
livro, inclusive mantendo a notação e numeração que são usados pelo autor.
Além disso, vamos incluir observações, soluções para os exercícios, e justificar
passos que ficaram em segundo plano na abordagem do livro.

\setcounter{section}{2}
\section{Árvores}
\begin{definition}
  Uma \emph{árvore} é um conjunto ordenado \((T,\leq)\) tal que:
 \begin{enumerate}[(i)]
  \item \(T\) possui um menor elemento;
  \item para cada \(x\in T\), o conjunto \(\{y\in T\,\colon y<x\}\) é bem
    ordenado sob \(\leq\).
  \end{enumerate}
\end{definition}
Os elementos de \(T\) são chamados de \emph{nós}. Em particular,
o menor elemento de \(T\) é chamado  de \emph{raíz}.
Se \(x,y\in T\) são nós tais que  \(y<x\), dizemos que \(y\) é um
\emph{antecessor} de \(x\), e que \(x\) é um \emph{sucessor} de \(y\).

Seja \(x\in T\) um nó.  A \emph{altura} \(h(x)\) de \(x\) é o único ordinal
isomorfo ao conjunto bem ordenado \mbox{\(\{y\in T\,\colon
y<x\}\),} composto por todos os antecessores de \(x\). Observe que o Teorema 3.1
do Capítulo 6 garante que a altura está bem-definida como função em \(T\).
Além disso, se \(h(x)\) é um ordinal sucessor, dizemos que \(x\) é um
\emph{nó sucessor}. Caso contrário \(x\) é denominado \emph{nó limite}.
O \(\alpha\)-ésimo \emph{nível} de \(T\) é o conjunto
\(T_\alpha=\{x\in T \,\colon h(x)=\alpha\}\). A \emph{altura} \(h(T)\) da árvore
\(T\) é o menor ordinal \(\alpha\) tal que \(T_\alpha=\varnothing\).  

Um \emph{ramo} \(b\) de \(T\) é uma cadeia maximal em \(T\).  O \emph{comprimento}
\(\ell(b)\) de um ramo \(b\) é o tipo da ordem de \(b\). Note que para
todo ramo \(b\) de \(T\) temos \(\ell (b)\leq h(T)\).
Um ramo cujo comprimento é igual a \(h(T)\) é chamado de \emph{cofinal}.
Uma \emph{subárvore} \(T^\prime\) de \(T\) é um subconjunto
\(T^\prime\subseteq T\) tal que, para quaisquer \(x\in T^\prime\) e \(y\in T\),
temos que \(y<x\) implica \(y\in T^\prime\).
Sendo assim, \(T^\prime\) também é uma árvore quando ordenada por \(\leq\).
Ademais, para cada \(\alpha < h(T^\prime)\) temos que o \(\alpha\)-ésimo nível
de \(T^\prime\) é dado por \(T_\alpha^\prime= T_\alpha \cap T^\prime\).

 Para cada \(\alpha\leq h(T)\), o conjunto \(T^{(\alpha)}=\bigcup_{\beta <
   \alpha} T_\beta\) é uma subárvore de \(T\) com  \(h(T^{(\alpha)})=\alpha\).
 Se \(x\in T_\alpha\) então \(\{y\in T\,\colon y < x\}\) é um ramo de \(T^{(\alpha)}\) de
comprimento \(\alpha\); entretanto, se \(\alpha\) é um ordinal limite então
\(T^{(\alpha)}\) pode ter outros ramos de comprimento \(\alpha\).

Finalmente, um conjunto \(A\subseteq T\) é uma \emph{anticadeia} em \(T\) se
quaisquer elementos de \(A\) são incomparáveis. Isto é, se \(x,y\in A\) são tais
que \(x\not = y\), então \(x\not \leq y\) e \(y\not \leq x\).

A respeito dos conceitos introduzidos acima, é importante observar que todo
ramo e subárvore de \(T\) contém a raiz. Também notamos que se \(x,y\in T\)
são nós tais que \(y< x\), então \(h(y) <h(x)\). Além disso, a prórpia definição
de \(h(T)\) nos dá que \(T_\alpha\not=\varnothing\) para todo ordinal
\(\alpha<h(T)\). A seguir, vamos resolver os
exercícios sugeridos pelo autor neste ponto do texto em \cite{jech},
que exploram algumas das propriedades que surgem imediatamente das
definições que apresentamos:

\begin{exercicio}
  Seja \((T,\leq)\) uma árvore. Mostre que:
  \begin{enumerate}[(i)]
  \item O único elemento \(r \in T\) tal que \(h(r)=0\) é a raíz. Em particular,
    \(T_0\not =\varnothing\);
  \item Se \(\alpha,\beta\) são ordinais tais que \(\alpha\not = \beta\), então \(T_\alpha\cap T_\beta =\varnothing\);
  \item \(T=T^{(h(T))}=\bigcup_{\alpha < h(T)}T_\alpha\);
  \item Um nó \(x\in T\) é um nó sucessor se, e somente se existe um único nó \(y\in T\) tal que
    \begin{equation}\label{propsuc}
      y < x \text{ e não existe } z \text{ tal que } y<z<x.
    \end{equation}
    Se \(x\) é um nó sucessor então o nó
    \(y\) que satisfaz \eqref{propsuc}  é chamado de \emph{antecessor imediato} de
    \(x\), e dizemos que \(x\) é \emph{sucessor imediato} de \(y\).
    Note que cada nó sucessor possui um único antecessor imediato,
    mas pode possuir múltiplos sucessores imediatos;
    \item Se \(x,y\in T\) são nós tais que \(y< x\), então existe um único \(z\in T\) tal que \(y< z\leq x\) e
      \(z\) é um sucessor imediato de \(y\);
    \item \(h(T)=\sup\{\alpha +1\,\colon
      T_\alpha\not=\varnothing\}=\sup\{h(x)+1\,\colon x\in T\}.\)
    \end{enumerate}
  \end{exercicio}
  
\begin{proof}[Solução]\hfill
  \begin{enumerate}[(i)]
  \item Por definição, \(T\) possui um menor
    elemento, o que implica que \(T\) possui uma raíz \(r\). Para verificar a
    unicidade de \(r\), suponha que \(T\) possui duas raízes \(r_1\) e \(r_2\)
    distintas. Neste caso,  para todo \(x\in T\) temos que
    \(r_1\leq x\). Em particular temos que
    \(r_1\leq r_2\). Analogamente, obtemos que \(r_2\leq r_1\). Como \(\leq\) é
    uma ordem, segue que \(r_1=r_2\), contradição. Finalmente notamos que,
    como \(r\) é o menor elemento de \(T\), segue que \(\{x\in T\,\colon
    x<r\}=\varnothing\). Como último fato implica que \(h(r)=0\), concluímos que
    \(r\in T_0\). Logo, \(T_0\not=\varnothing\).
  \item Suponha que existe \(x\in T_\alpha\cap T_\beta\). Neste caso, temos que
    \(\alpha=h(x)=\beta\), o que contradiz a unicidade de \(h(x)\).
  \item Para a primeira igualdade, note que \(T^{(h(T))}\subseteq T\) uma vez
    que \(T^{(h(T))}\) é definido como uma união de subconjuntos de \(T\). Por
    outro lado, seja \(x\in T\) e suponha que \(x\not\in T^{(h(T))}\). Daí,
    segue que não existe \(\alpha < h(T)\) tal que \(h(x)=\alpha\). Isso implica
    que \(h(x)\geq h(T)\), absurdo pois \(T_\beta=\varnothing\) para todo
    ordinal \(\beta \geq h(T)\). Como a segunda igualdade segue diretamente da
    definição de \(T^{(h(T))}\), a prova está completa. 
  \item Suponha que \(x\) é um nó sucessor. Neste caso, temos por definição
    que \(h(x)=S(\alpha)\) para algum ordinal \(\alpha\). Vamos mostrar que
    \(T_\alpha\) é não vazio. Assumindo o contrário, teremos que
    \(h(x)>\alpha\geq h(T)\), que é um absurdo. Além disso,
    se  existe \(z\) tal que \(y<z<x\), então
    \[h(y)=\alpha<h(z)<S(\alpha)=h(x),\] absurdo. Finalmente, suponha que
    existem \(y_1\)
    e \(y_2\) distintos tais que \(h(y_1)=h(y_2)=\alpha\), neste caso o conjunto \(S=\{w\in
    T\,\colon w <x\}\) não é bem ordenado pois o subconjunto
    \(\{y_1,y_2\}\not = \varnothing\) de \(S\) não possui um menor elemento.
    Para a volta, basta mostrar que se existe \(y\) satisfazendo
    \eqref{propsuc}, então \(h(x)=S(h(y))\).
  \item Em primeiro lugar, suponha por absurdo que não existe tal elemento
    \(z\). Seja \(\alpha=h(x)\) e considere a subárvore \(T^{(\alpha)}\) de \(T\).
    Seja \(b=\{w \in T \,\colon w<x\}\) e recorde que \(b\) ramo de
    \(T^{(\alpha)}\). Note que \(y\in b\) e que \(b\) tem comprimento \(\alpha\). Se
    não existe \(z\), então \(b\) não é maximal.
  \item Primeiramente, lembramos que por definição, \(h(T)\) é o menor ordinal
    \(\alpha\) tal que \(T_\alpha=\varnothing\). Sendo assim, podemos ver que
    \[h(T)\geq \alpha +1 \text{ para cada } \alpha \text{ tal que }
      T_\alpha\not=\varnothing,\]
    e que
    \[h(T)\geq h(x)+1\,\colon x\in T.\]
    Daí segue que
    \[h(T)\geq \sup\{\alpha+1\,\colon T_\alpha\not=\varnothing\}\text{ e que }
      h(T)\geq\sup\{h(x)+1\,\colon x\in T\}.\]
    Por fim, suponha que
    \[h(T)> \sup\{\alpha+1\,\colon T_\alpha\not=\varnothing\}\text{ ou } h(T)
      >\sup\{h(x)+1\,\colon x\in T\}.\]
    Em ambos os casos, segue que \(T_\beta=\varnothing\) para algum \(\beta<
    h(T)\). Absurdo.\qedhere
\end{enumerate}
\end{proof}
\begin{exercicio}
    Mostre que:
    \begin{enumerate}[(i)]
    \item Cada cadeia em \(T\) é bem-ordenada;
    \item Se \(b\) é um ramo de \(T\) e \(x\in b\), e \(y<x\), então \(y\in b\);
    \item Se \(b\) é um ramo em \(T\), então \(\card{b\cap T_\alpha}=1\) para
      \(\alpha <\ell (b)\) e \(\card{b\cap T_\alpha}=0\) para \(\alpha
      >\ell (b)\). Conclua que \(\ell (b)\leq h(T)\);
    \item \(h(T)=\sup\{\ell (b)\,\colon b \text{ é um ramo de } T\}\);
    \item \(T_\alpha\) é uma anticadeia para cada \(\alpha < h(T)\).
     \end{enumerate}
  \end{exercicio}
  \begin{proof}[Solução]\hfill
    \begin{enumerate}[(i)]
      \item  Seja \(C\subseteq T\) uma cadeia e seja
        \(\varnothing\not = S\subseteq C\) vamos provar que \(S\) possui um menor
        elemento. Com efeito, para cada \(y\in S\), existe \(x\in S\) tal que \(y >x\), logo fixado $y_0\in S$ qualquer o conjunto $A_{y_0} =\{z\in S: z<y_{0}\}$ é não vazio e portanto admite um menor elemento, pois está contido em $\{z\in T: z<y\}$ que é bem ordenado por definição. Seja $m$ o menor elemento de $A_{y_0}$, vou provar que $m$ é o menor elemento de $S$, de fato, se $y\in S$ então ou $y\geq y_0$ ou $y<y_0$, pois $S\subset C$ que é uma cadeia.
        Portanto se $y\geq y_0$ então $m<y_0\leq y$, e por outro lado se $y<y_0$ então $y\in A_{y_0}$ e assim $m\leq y$. De toda forma $m\leq y$ e por fim $m$ é o menor elemento de $S$.
      \item Por definição, um ramo é bem-ordenado e maximal...
      \item Suponha que \(y\not\in b\). Vamos mostrar que \(y\) é comparável
        a todos os elementos de \(b\). Se \(z\in b\) e \(z < x\), temos que
        \(y\) e \(z\) são comparáveis pois ambos pertencem ao conjunto \(\{w\in
        T\,\colon w< x \}\), que é bem ordenado, e em particular, linearmente
        ordenado. Caso \(x<z\), obtemos que \(y<z\) por transitividade. Assim,
        concluímos que \(b\cup\{y\}\) é uma cadeia e portanto \(b\) não é
        maximal. Absurdo.
      \item
        \item Seja \(\alpha < h(T)\) e assuma que existem \(x,y\in T_\alpha\)
          distintos tal que \(x\) e \(y\) são comparáveis.
          Suponha sem perda de generalidade que
      \(x<y\). Neste caso, temos que \(x<y\) mas \(h(x)=h(y)\), absurdo.\qedhere
  \end{enumerate}
  \end{proof}

  Agora vamos olhar para alguns exemplos de árvores:
  \begin{exemplo}\hfill
    \begin{enumerate}[(a)]
    \item Todo conjunto bem-ordenado \((W,\leq)\) é uma árvore. Sendo assim,
      podemos pensar em árvores como generalizações de boas ordens. A altura
      \(h(W)\) é o tipo de ordem de \(W\), o único ramo de \(W\) é o próprio
      \(W\), que é cofinal.
    \item Seja \(\lambda\) um número ordinal e seja \(A\) um conjunto não-vazio.
      Defina \(A^{< \lambda}=\bigcup_{\alpha < \lambda} A^\alpha\) como o
      conjunto de todas as sequências transfinitas de elementos de \(A\) com
      comprimento menor que \(\lambda\). Considere \(T=A^{< \lambda}\) e o
      ordene segundo \(\subseteq\), de modo para \(f,g\in T\), temos \(f\leq g\)
      se, e somente se \(f\subseteq g\). Isto é, \(f=\rest{g}{\dom(f)}\). É fácil
      verificar que \(T\) é uma árvore. Para \(f\in T\),\(h(f)=\alpha \iff f\in A_\alpha\),
      isto é, \(T_\alpha =A^\alpha\). Para \(\alpha =\beta +1\) e \(f\in A^\alpha\),
      \(\rest{f}{\beta}\) é o antecessor imediato de \(f\), e \(f\cup {\langle \beta ,\alpha \rangle}\)
      são sucessores imediatos de \(f\) para todo \(\alpha \in A\)
      Ramos em \(T\) estão em correspondência 1-a-1 com as funções de \(\lambda\) em \(A\):
      se \(F\in A^{< \lambda}\) então \({\rest{F}{\alpha}\| \alpha <\lambda}\) é um ramo em \(T\).
      Por outro lado, se \(B\) é um ramo em \(T\) então \(B\) é um sistema compatível de funções
      e \(F=\bigcup B\in A^\lambda\). Notemos que todos os ramos são cofinais.
    \item Generalizando, se \(T\subseteq A^{<\lambda}\) for uma subárvore de 
    \((A^{<\lambda},\subseteq)\), então os ramos em \(T\) estão em correspondência
    1-a-1 com as funções \(F\in A^{<\lambda}\cup A^\lambda\) para as quais temos
    \(\rest{F}{\alpha}\in T\) para todo \(\alpha \in \dom(F)\), e ou \(F\notin T\),
    ou \(F\in T\) e \(F\) não possui sucessores em \(T\). Nessa situação, costuma-se
    identificar ramos com suas funções correspondentes
    \item Seja \(A=\Naturals \),\(\lambda =\omega\). Consideremos \(T\subseteq \Naturals^{<\omega}\)
    o conjunto das sequências finitas decrescentes, ou seja, \(f\in T \iff f(i)>f(j)\) 
    para todo \(i<j<\dom(f)\in\Naturals\), então \(T\) é uma subárvore de \((\Naturals^{<\omega},\subseteq)\).
    T é uma árvore de altura \(\omega\) que não possui ramos cofinais (veja Exercício 2.8,
    Capítulo 3).
    \item Seja \((\Reals,\leq)\). Uma \emph{representação} de uma árvore \((T,\preceq)\)
    \emph{por intervalos} em \((\Reals,\leq)\) é uma função 1-a-1 \(\Phi\) que para cada 
    \(x\in T\), \(\Phi (x)\)é um intervalo em \((\Reals,\leq)\) de modo que para todo \(x,y\in T\):
    \begin{enumerate}[(i)]
    \item \(x\preceq y\iff \Phi(x)\supseteq\Phi(y)\); 
    \item \(x\) e \(y\) são incomparáveis \(\iff\) \(\Phi(x)\cap\Phi(y) =\emptyset\)
    \end{enumerate}
    Isso implica, em particular, que \((\Phi[T],\supseteq)\) é uma árvore isomorfa à
    \((T,\preceq)\). Por exemplo, o sistema \(\langle D_s \| s\in S\rangle\) cosntruído
    no Exemplo 3.18, Capítulo 10, é uma representação da árvore \(S=Seq({0,1})={0,1}^{<\omega}\)
    (ordenado por \(\subseteq\) por intervalos fechados na reta real.
    \end{enumerate}
  \end{exemplo}
  
  
  O estudo de árvores finitas é um dos conceitos principais de combinatória. Nós não vamos
  nos aprofundar nisso aqui, tomaremos nossa atenção para árvores infinitas. Nos preocuparemos
  majoritariamente com sobre quais circunstâncias uma árvore possui ramo cofinal.
  Para árvores cuja altura é um ordinal sucessor a resposta é óbvia: se \(h(T)=\alpha +1\),
  então \(T_\alpha\neq\emptyset\) e \({y\in T\| y\leq x}\) é ramo cofinal em \(T\) para qualquer
  \(x\in T_\alpha\). Assim, focaremos em árvores de altura limite. O Exemplo 3.2(d) nos diz que
  existem árvores de altura \(\omega\)que possuem somente ramos finitos. O próximo teorema, a
  observação mais básica relevante para nossos estudos, nos mostra que isso não pode acontecer se
  a árvore é suficientemente "esguia"

    \begin{teo}[Lema de König]
  Se $T$ é uma árvore de altura $\omega$, com todos os níveis finitos, então T admite um ramo de altura $\omega$.
  \end{teo}
  \begin{proof}
  Equivalentemente, temos que provar que, toda árvore de altura $\omega$, tal que cada nó tem um numero finito de sucessores imediatos, tem um ramo infinito.
  De fato vamos construir, com recursão, uma sequência(infinita) $\langle c_n\rangle_{n=0}^{\infty}$ de nós de $T$ tal que, para todo $n$, $\{a\in T: c_n\leq a\}$ é infinito. Pelo exercício \ref{ex31-iv}
  $$
  \{a\in T: c_n\leq a\} = \{c_n\} \cup\bigcup_{b\in S} \{a\in T: b\leq a\}
  $$
  onde $S$ é o conjunto dos sucessores imediatos de $c_n$. Logo, para pelo menos um $b\in S$, $\{a\in T: b\leq a\}$ é infinito, assim defina $c_{n+1}$ um tal $b$. 
  Falta verificar que $\{a\in T: c_n\geq a \text{ para algum } n\}$ é um ramo de $T$ com altura $\omega$.
  
  \end{proof}
  O próximo exercício será uma generalização do Lema de König.
  
  
  Então podemos considerar o seguinte problema: Se $T$ é uma árvore de altura $\omega_1$, onde cada nível é contável, é verdade que $T$ tem um ramo de comprimento $\omega_1$? NÃO!
\begin{definition}
  Uma árvore de altura $\omega_1$ é chamada de uma \textit{árvore de Aronszajn} se todos os os seus níveis são no máximo contáveis e se não possui ramos de comprimento $\omega_1$.
\end{definition}
  
\begin{theorem}
  Existem árvores de Aronszajn de altura $\omega_1$.
\end{theorem}

\begin{proof}
  Construímos os níveis $T_{\alphah}$, $\alpha < \omega_1$ de uma árvore de Aronszajn por recursão transfinita de tal modo que
  
  \begin{enumerate}
      \item $T_{\alpha} \subseteq \omega^{\alpha}$; $\left|T_{\alpha}\right| \le \aleph_0$;
      \item Se $f \in T_{\alpha}$, então $f$ é um-pra-um e $(\omega - ranf)$ é infinito;
      \item Se $f \in T_{\alpha}$ e $\beta < \alpha$ então $f |_{\beta} \in T_{\beta}$;
      \item Para qualquer $\beta < \alpha$, qualquer $g \in T_{\beta}$, e qualquer $X \subseteq \omega - rang$ finito, existe uma $f \in T_{\alpha}$ tall que $f \supseteq g$ e $ranf \cap X = \emptyset$.
  \end{enumerate}
  Vamos assumir que isso foi feito, e vamos mostrar que $T = \bigcup_{\alpha < \omega_1}T_{\alpha}$ é uma árvore de Aronszajn. Claramente $T$ é uma árvore por $(3)$, cada nível é no máximo contável por $(1)$, e sua altura é $\omega_1$ (por $(4)$, cada $T_{\alpha} \neq \emptyset$). Se $B$ fosse um ramo de comprimento $\omega_1$ em $T$, então $F = \bigcup B$ seria uma função um-pra-um de $\omega_1$ para $\omega$ (por $(2)$), uma contradição.
  
  Falta construir $T_{\alpha}$ para $\alpha$ limite. Para qualquer $g \in T_{\beta}$, $\beta < \alpha$, e qualquer $X \subseteq \omega - rang$ finito, construímos uma $f = f(g, X)$ recursivamente da seguinte maneira. Fixe uma sequência crescente $\langle \alpha_n \rangle_{n=0}^{\infty}$ tal que $\alpha_0 = \beta$ e $sup\{\alpha_n | n \in \mathbb{N}\} = \alpha$. Seja $f_0 = g \in T_{\alpha_0}$ e $X_0 = X \subseteq \omega - ranf_0$. Tendo definido $f_n \in T_{\alpha_n}$ e $X_n = X \subseteq \omega - ranf_n$ finitos, nós primeiro tomamos um $X_{n+1} \supset X_n$ finito, $X_{n+1} = X \subseteq \omega - ranf_n$ (isto é possível porque o último conjunto é infinito, por $(2)$) e então selecionamos algumas $f_{n+1} \in T_{\alpha_{n+1}}$ tais que $f_{n+1} \supseteq f_n$ e $X_{n+1} \cap ranf_{n+1} = \emptyset$ (possível por $(4)$). Seja $f = \bigcup_{n = 0}^{\infty}f_n$. Claramente $f: \alpha \rightarrow \omega$, $f$ é um-pra-um (pois todas as $f_n$ são), $ranf \cap (\bigcup_{n = 0}^{\infty} X_n) = \emptyset$, portanto $\omega - ranf$ é infinito, e $ranf \cap X = \emptyset$. Então $f$ satisfaz $(2)$. Para $\beta < \alpha$, $f |_{\beta} = f_{n} |_{\beta}$ quando $\beta < \alpha_n$, e portanto $(3)$ também é satisfeito.
  
  Colocamos esta $f = f(g, X)$ em $T_{\alpha}$ para cada $g \in \bigcup_{\beta < \alpha} T_{\beta}$ e cada $X \subseteq \omega - rang$. Portanto $(4)$ também é satisfeito. Como $|\bigcup_{\beta<\alpha}T_{\beta}| \le \sum_{\beta<\alpha}|T_{\beta}| \le \aleph_0$ (pela suposição indutiva $(1)$) e o número de subconjuntos finitos de $\omega$ é contável, o conjunto $T_{\apha}$ é no máximo contável, e $(1)$ também é satisfeito.
\end{proof}

Mais geralmente, uma árvore de altura $\kappa$ ($\kappa$ um cardinal não-contável) é chamada de uma árvore de Aronszajn se todos os seus níveis tem cardinalidade menor que $\kappa$ e não existem ramos de comprimento $\kappa$. A questão da existencia de tais árvores é muito complicada e ainda não foi totalmente resolvida. É fácil mostrar que elas sempre existem quando $\kappa$ é singular (exercício). São de interesse particular os cardinais não-contáveis para os quais vale um análogo do Lema de Kőnig, i.e., não existem árvores de Aronszajn de altura $\kappa$; dizemos que tais cardinais possuem a \textit{propriedade da árvore}. Acontece que cardinais fortemente inacessíveis com a propriedade da árvore são precisamente os cardinais fracamente compactos definidos na seção 2.
  
  \bibliographystyle{plain}
  \bibliography{sample}
\end{document}
